\documentclass[12pt]{article}
\usepackage[top=0.5in, bottom=0.5in, left=0.5in, right=0.5in]{geometry}
\usepackage{parskip}
\setlength{\parindent}{0pt}
\renewcommand{\familydefault}{\sfdefault}

\begin{document}

\section*{Lesson Script: Identifying the Domain as Interface (30-minute Demo)}

\textbf{Setting:} 11th/12th grade calculus students at a consortium high school, actively working on a project. You are a visiting teacher candidate leading a 30-minute mini-lesson framed as a mutual fit check.

\textbf{Goal:} Establish domain as the part of a function that lives on the boundary between the real world and the abstract model — the part the user controls.

\section*{Full Lesson Monologue: Domain as Interface}

Good morning. Or hello. Or whatever level of greeting this class tolerates at this hour.

Tomorrow I might be your teacher, or I might not. I’m here to teach a demo lesson, and I want to be clear: we’re interviewing \textit{each other}. This is called a fit.

I need to see if I can teach you something you want to receive. You need to see if you can pick up what I’m putting down. And if we don’t match? No harm. You’ll probably never see me again. But for the next few minutes—give me your eyes.

You’re not just here to solve for \( x \). You're here to learn how to \textbf{think}. Because your mind—particularly yours—is a battleground.

We are surrounded by tools that simulate thinking. They generate answers, predict outcomes, even write code. And many people have started confusing this with actual thinking. They believe thinking is being \textit{replaced}.

That is a mistake.

What’s happening is: thinking has been placed at the very center of society. The ability to \textit{understand}, to \textit{interrogate}, to \textit{model} a system—that’s more important than ever. If you can’t do that, you’ll spend your life being handed outputs and never knowing how they came to be.

So today we’re going to talk about \textbf{functions}. Specifically: what is a function’s \textbf{domain}?

And I know. You’ve seen this before. It’s the thing you write in interval notation. But today I want you to look at it differently. I want you to see it as an \textit{interface}. As a boundary.

Imagine a function is a machine. An app. You give it input. It does some hidden thing inside. It gives you an output.

That input? That’s the \textbf{domain}. It’s the only part you get to touch. The only piece you control. Everything else—the guts of the system, the rule, the constants, the coefficients—that’s on the inside. You can’t touch that unless you’re designing the function.

This matters. Because the domain is the only part where \textit{you} exist. If you don’t know what the domain is, you don’t know what your job is.

So when we talk about domain today, we are not just asking ‘what values can \( x \) take?’ We’re asking: \textit{where is the system connected to the real world?} What are the things that can vary? What’s the source of change? That’s your domain.

Not everything that’s in the function is part of the domain. That confuses people. Especially in real-world scenarios. So your job today is to be clear: what’s on the outside? What is being \textit{given} to the function?

Because math is just a language we use to describe systems. But the first step is understanding which part of the system is under our control.

Let’s find that boundary. Let’s locate the interface. That’s the point of this lesson. Everything else is details.

\textbf{Script (as spoken):}

\textit{(Step into the room, warm but focused)}

"Hi everyone. I’m Eric Allatta. I go by Eric, Allatta, and Mister. I’ll be with you briefly today. I’ve been invited to teach a short demo lesson — but really this is an interview going both ways. I’m here to figure out if this school is a good fit for me, and you’re here to figure out if I’m someone who belongs in front of you."

"You might not see me again after today. And I’m not here to waste your time. So I’m not going to walk you through a long lesson or dump a million problems on your desk. We’re going to do something short and sharp — and if it hits, we keep going. If not, you can go back to what you were doing. Deal?"

\textit{(Pause — short beat — begin framing)}

"Today’s topic is: domain."

"Yes, I know — the thing you label on a graph, or the first number in a function. But today we’re going to think about it differently. Not as a number set, but as a \textbf{boundary}. A kind of \textit{interface}."

"Let me explain."

"Imagine a function is a machine — or an app. Something that takes an input, does some work, and gives you an output. Most of the time, you don’t see the guts of the system. You don’t see the rules. You just know: I put this in, I get that out."

"That input — what you get to change — \textbf{that’s the domain}."

"The domain isn’t what the function knows. It’s what \textit{you} know. It’s the real-world part. The part that lives \textbf{outside} the system. And the only way into the system is through that domain. It’s the only handle you’ve got."

\textit{(Shift tone — now guiding them)}

"So when you see a function scenario, the first question is: \textit{what do I control? What can I vary? What’s on the outside?}"

"Let’s look at an example together."

\textit{(Use a whiteboard or projector. Example: A YouTube video gains 500 views per hour. Prompt the class.)}

"In this scenario — what’s changing? What do we control?"

\textit{(Wait for students to say \textit{time}, or prompt them if needed.)}

"Yes — time. Time is the domain. It’s what we can change. The views? That’s the output. The rule that says 500 views per hour? That’s the inside of the system. That’s what happens \textit{after} you’ve made a choice about time."

"This distinction is important. The domain isn’t ‘what’s mentioned in the problem.’ It’s \textit{what drives it forward}. The interface."

\textit{(Transition to pair work)}

"I’m giving you a sheet of quick scenarios. Work with someone nearby. For each one, I want you to do three things: \textbf{Underline what’s changing}, \textbf{circle what comes out}, and then \textbf{write one sentence} answering: ‘What is the domain in this case?’"

"Don’t worry about getting them all done. Quality over quantity."

\textit{(During work time: walk around, ask questions like “What’s the driver here?” or “Do you get to control that part?”)}

\textbf{[~10 minutes of pair work]} 

\textit{(Pull them back together)}

"Let’s share a few. Just shout out a scenario — what did you pick as the domain?"

\textit{(Get a few responses. React in real time, correcting gently if needed.}

"Exactly — the domain is the set of inputs we feed the machine. Everything else? It reacts. It’s not yours to change."

\textit{(Stretch activity, if time permits)}

"If you finish early — or if your group is feeling ambitious — try this: make up your own scenario. It can be totally real or totally made-up. But it has to have an input, an internal system, and an output. And it should be clear where the domain is — where the interface is."

"That’s how we model the world — by identifying what we can control. Everything else? We have to live with."

\textit{(Wrap up)}

"Thanks for thinking with me. If we don’t get to keep working together — I hope this helped you see domain in a new way. If we do? Get ready for more."

\end{document}
