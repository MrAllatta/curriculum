\documentclass[12pt]{article}
\usepackage[top=0.5in, bottom=0.5in, left=0.5in, right=0.5in]{geometry}
\usepackage{parskip}
\setlength{\parindent}{0pt}
\renewcommand{\familydefault}{\sfdefault}
\usepackage{enumitem}
\usepackage{amsmath}

% ---------- BEGIN LESSON PLAN ----------

\begin{document}

\textbf{Lesson Title:} Identifying Domain Through Abstraction and Real-World Modeling \\
\textbf{Grade Level:} 11th/12th Grade Calculus \\
\textbf{Duration:} 30 minutes (core lesson) \\
\textbf{Prepared By:} Eric Allatta \\
\textbf{Date:} \today \\

\textbf{Lesson Overview:} \\
This lesson challenges students to identify and construct the domain of a function in real-world contexts, using the lens of abstraction from computer science. Emphasis is placed on distinguishing between external inputs and internal logic—understanding the function interface. Students engage in scenario analysis and creation, constructing their own models while surfacing the conceptual distinction between what drives change (domain) and what is embedded within the system’s rules (function logic).

\textbf{Learning Objectives:}
\begin{itemize}[nosep]
  \item Define domain in terms of function inputs within a real-world context.
  \item Distinguish between external inputs (domain) and internal logic or constants within the function.
  \item Construct and represent original function-based scenarios, explicitly identifying domain, rule, and output.
  \item Articulate the boundary between real-world data and abstract mathematical modeling.
\end{itemize}

\textbf{Materials Needed:}
\begin{itemize}[nosep]
  \item Printed copies of scaffolded domain examples (from existing packet)
  \item Whiteboard or projector
  \item Student worksheet: \textit{Build-A-Function Scenario Creator}
  \item Optional: Sticky notes for gallery walk feedback
\end{itemize}

\textbf{Vocabulary:} Domain, Range, Function, Input, Output, Interface, Abstraction, System

\textbf{Lesson Sequence:}

\textbf{1. Opening Framework (5 minutes)}
\begin{itemize}[nosep]
  \item Deliver intro monologue (adjusted for tone): Explain that this is a demo lesson and a mutual fit process.
  \item Use the metaphor of the function as a machine or app: what goes in, what happens inside, and what comes out.
  \item Pose a question: \textit{"What can you control in this situation? What is already programmed?"}
\end{itemize}

\textbf{2. Scaffolded Analysis: Identifying Domain (10 minutes)}
\begin{itemize}[nosep]
  \item Distribute pre-written scenarios.
  \item Students work in pairs to annotate: Identify what is changing (domain), what is inside the function (rule), and what comes out (range).
  \item Brief share-out: Collect examples of domains on board to show variety.
\end{itemize}

\textbf{3. Mini-Direct: Domain as Interface (5 minutes)}
\begin{itemize}[nosep]
  \item Clarify the difference between input data and internal logic.
  \item Use computer science framing: domain is the interface point between the user and the system.
  \item Reinforce: domain is not just the variable, it's the driver of change—what the user controls.
\end{itemize}

\textbf{4. Student Scenario Creation (Extension if time permits)}
\begin{itemize}[nosep]
  \item Students develop original real-world function scenarios.
  \item Prompts:
    \begin{itemize}[nosep]
      \item Define a change someone can control (domain)
      \item Describe what happens inside the system (function logic)
      \item Identify what is produced (output)
      \item Predict the function shape (linear, exponential, piecewise, etc.)
    \end{itemize}
  \item Optional challenge: Create a "function trap"—a scenario with misleading domain cues.
\end{itemize}

\textbf{Assessment:}
\begin{itemize}[nosep]
  \item Formative: Observation of student discussion and annotation during scenario analysis.
  \item Artifact: Student-created scenario with labeled domain, rule, and output (if completed).
  \item Extension: Optional written reflection: \textit{"What was the hardest part about identifying domain today?"}
\end{itemize}

\textbf{Differentiation:}
\begin{itemize}[nosep]
  \item Scaffolded examples for students needing structure.
  \item Open-ended creation and abstraction work for students ready to stretch.
  \item Peer collaboration to support wide skill range in open enrollment context.
\end{itemize}

\textbf{Extension / Homework:}
\begin{itemize}[nosep]
  \item Identify a function in your own life. What’s the domain? What happens inside the system? What’s the output?
  \item Bring one example to next class and be ready to explain its domain.
\end{itemize}

\end{document}
