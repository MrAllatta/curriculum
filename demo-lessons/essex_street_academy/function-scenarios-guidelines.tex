\documentclass[12pt]{article}
\usepackage[top=0.5in, bottom=0.5in, left=0.5in, right=0.5in]{geometry}
\usepackage{parskip}
\setlength{\parindent}{0pt}
\usepackage{enumitem}
\setlength{\parindent}{0pt}
\renewcommand{\familydefault}{\sfdefault}

\begin{document}

\section*{Guidelines for Creating a Function Scenario}

\textbf{Purpose:} You're not just writing a math problem. You're describing a real-world situation where something changes — and where that change can be modeled by a function.

\textbf{1. Find the Data in Your World}

Data is everywhere. The moment something is counted or described in a structured way, it becomes data.

\begin{itemize}[nosep]
  \item Data can be \textbf{quantitative} (measured or counted) or \textbf{qualitative} (described).
  \item Data sources include sensors, surveys, scanners, and human observation.
  \item Schools are data factories — they collect, generate, and transform information daily.
\end{itemize}

\textit{Ask: What is being tracked, measured, or observed in this setting?}

\textbf{2. Choose a Quantity to Focus On}

Pick a variable that changes — something someone might control or observe. This becomes your \textbf{input}, or \textbf{domain}.

\begin{itemize}[nosep]
  \item Examples: time, money, distance, number of people, number of items
  \item This is what drives the situation forward
\end{itemize}

\textbf{3. Ask: What Depends on That Input?}

This is your \textbf{output}. It’s what the function helps you find once you know the input.

\begin{itemize}[nosep]
  \item What changes when your input changes?
  \item What is being calculated or predicted?
\end{itemize}

\textbf{4. Think About the Purpose}

Why does this function exist? Who uses it?

\begin{itemize}[nosep]
  \item A shopper trying to budget?
  \item A system predicting behavior?
  \item A business tracking output?
  \item A person making a decision?
\end{itemize}

\textit{Functions are not just equations. They're tools for thinking about problems.}

\textbf{5. Predict the Function Type}

What kind of relationship fits your scenario?

\begin{itemize}[nosep]
  \item \textbf{Linear}: constant rate, steady change
  \item \textbf{Quadratic}: has a curve or turning point
  \item \textbf{Exponential}: rapid increase or decay
  \item \textbf{Piecewise}: different rules in different cases
  \item \textbf{Other}: something more complex or unique
\end{itemize}

\textbf{In Short:}
\begin{enumerate}[nosep]
  \item Start with something real.
  \item Identify what you can measure or control — that's the domain.
  \item Describe what depends on that input — that's the output.
  \item State why someone would use this function.
  \item Take a guess at what kind of function fits.
\end{enumerate}

\end{document}
