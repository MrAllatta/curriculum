\documentclass[12pt]{article}
\usepackage{enumitem}
\usepackage{hyperref}
\usepackage[margin=0.4in]{geometry}
\usepackage{setspace}
\usepackage{titlesec}
\usepackage{ulem} % for underline
\usepackage{amsmath}
\usepackage{helvet}
\renewcommand{\familydefault}{\sfdefault} % Set sans-serif as default
\titlespacing*{\section}{0pt}{0.5em}{0.25em}

\titleformat{\section}{\large\bfseries}{}{0em}{}

\begin{document}
\noindent\textbf{Contract Detective: What's the Contract?} \textit{Computing in the News} \hfill \\  \textbf{Name: \underline{\hspace{3in}} Date: \underline{\hspace{1in}}}

\setlist[itemize]{itemsep=0pt, topsep=0pt, left=1em}
\noindent\textbf{Directions:} Choose three innovations. Underline the \textbf{input}, circle the \textbf{output}, and bracket the \textbf{purpose}. After marking up each abstract, complete the contract handout in your own words.

\vspace{1em}
\section*{Computing Innovations}
\subsection*{Mild}
\begin{enumerate}[label=\arabic*.]
  \item \textit{Associated Press (2024)} -- \href{https://technews.acm.org/archives.cfm?fo=2024-05-may#15}{\textbf{States Turn to AI to Spot Guns at Schools}}\\
\footnotesize\textbf{Abstract:} Several U.S. states are considering or enacting programs to fund AI-powered surveillance systems that automatically detect people carrying firearms in school camera feeds. A prominent example is ZeroEyes, a computer-vision system that analyzes live video for visible guns and alerts authorities within seconds. In Kansas, a pending bill was written so narrowly (requiring the AI to be patented and already used in over 30 states) that only ZeroEyes meets the criteria, raising questions about lobbying. The goal is to enhance school safety by identifying threats faster than human staff could.\\

 \item \textit{The Wall Street Journal (2025)} -- \href{https://technews.acm.org/archives.cfm?fo=2025-03-mar#33}{\textbf{Smart Cameras Spot Wildfires Before They Spread}}\\
\footnotesize\textbf{Abstract:} The ALERTCalifornia project deploys a network of over 1,150 mountaintop cameras paired with AI 'digital lookouts' to catch wildfires early. The AI scans live feeds from fire-prone areas and has detected more than 1,200 fires, sometimes alerting authorities faster than 911 callers. Human operators then verify the blaze and dispatch firefighters. The system turns camera imagery into early warnings, demonstrating a direct input-output loop.\\

 \item \textit{Medical Xpress (2024)} -- \href{https://technews.acm.org/archives.cfm?fo=2024-05-may#16}{\textbf{BCI Decodes Words “Spoken” in the Brain in Real Time}}\\
\footnotesize\textbf{Abstract:} Researchers at Caltech developed a brain–computer interface (BCI) that can translate neural signals into words in real time. The system records neuron firing in speech-related brain regions and was trained on a small vocabulary. In tests, the BCI identified words the user was trying to say in their mind with up to 79% accuracy. It maps brain activity to language, turning electrical signals into computer-readable output.\\

 \item \textit{Fast Company (2024)} -- \href{https://technews.acm.org/archives.cfm?fo=2024-06-jun#21}{\textbf{Generative AI Scans Your Amazon Packages for Defects Before Shipment}}\\
\footnotesize\textbf{Abstract:} Amazon deployed an AI system called 'Project P.I.' to inspect packages at fulfillment centers. The system uses computer vision to analyze each box before it ships, checking for missing, incorrect, or defective items. It flags problems before delivery, improving accuracy and sustainability by reducing waste from incorrect shipments.\\

 \item \textit{University of Nottingham News (2024)} -- \href{https://technews.acm.org/archives.cfm?fo=2024-05-may#17}{\textbf{3D Printing Paves Way for Personalized Medication}}\\
\footnotesize\textbf{Abstract:} Researchers developed a 3D printing method that allows multiple medications to be combined into one pill. Using UV-sensitive materials, the printer creates structures that release drugs at controlled times. This makes it possible to tailor a pill's content and dosage to individual patients’ needs.\\

\end{enumerate}

\subsection*{Medium}
\begin{enumerate}[label=\arabic*.]

 \item \textit{IEEE Spectrum (2024)} -- \href{https://technews.acm.org/archives.cfm?fo=2024-05-may#48}{\textbf{New Techniques to Stop Audio Deepfakes}}\\
\footnotesize\textbf{Abstract:} In a U.S. contest to combat AI-generated voice “deepfakes,” researchers showcased multiple complementary innovations. The winning entries included OriginStory – a modified microphone that monitors a speaker’s physiological signals to verify a live human voice – and AI Detect – software that embeds machine learning into audio processing devices to spot signs of an artificial voice. Another solution, DeFake, adds subtle distortions to genuine recordings to prevent adversaries from cloning a voice.\\

 \item \textit{New Scientist (2024)} -- \href{https://technews.acm.org/archives.cfm?fo=2024-07-jul#27}{\textbf{Self-Replicating ‘Life’ Created from Digital ‘Primordial Soup’}}\\
\footnotesize\textbf{Abstract:} Google researchers produced artificial life by mixing random code snippets into a shared environment where they could recombine, mutate, and execute. Some aggregates began to self-replicate and compete for limited computational resources. Over time, these digital organisms evolved, with new variants outcompeting earlier ones.\\

\clearpage
\item \textit{Gizmodo (2024)} -- \href{https://technews.acm.org/archives.cfm?fo=2024-07-jul#29}{\textbf{Stores Roll Out AI-Powered Vending Machines That Sell Bullets}}\\
   \footnotesize\textbf{Abstract:} American Rounds is deploying vending machines that sell ammunition using facial recognition and ID scanning. Customers must scan their ID and face, which is matched using AI to verify their identity and age. If all checks pass, the machine dispenses bullets. The system replaces human clerks with an automated process integrating several layers of input verification.
 \\

 \item \textit{Cornell Chronicle (2025)} -- \href{https://technews.acm.org/archives.cfm?fo=2025-04-apr#39}{\textbf{‘Robotability Score’ Ranks NYC Streets for Robot Deployment}}\\
   \footnotesize\textbf{Abstract:} Cornell researchers developed a 'robotability score' to rank New York City streets based on how well they support robot navigation. The system uses data from NYC’s open datasets, along with image analysis from millions of street-level photos, to evaluate obstacles and crowding. This layered model generates a composite score indicating how suitable each street is for automated mobility solutions.

\end{enumerate}

\subsection*{Spicy}
\begin{enumerate}[label=\arabic*.]

 \item \textit{CNBC (2025)} -- \href{https://technews.acm.org/archives.cfm?fo=2025-04-apr#41}{\textbf{Eye-Scanning ID Project Launches in U.S.}}\\
\footnotesize\textbf{Abstract:} Worldcoin's World ID project is bringing biometric identity verification to the public through iris-scanning Orbs. The scan generates a cryptographic hash that proves a user is unique without storing their personal data. This credential can then be used to log into apps like Reddit or Minecraft to verify that the user is human, not a bot.\\

 \item \textit{Engadget (2025)} -- \href{https://technews.acm.org/archives.cfm?fo=2025-02-feb#35}{\textbf{MTA Used Google Pixels to Identify Subway Track Defects}}\\
\footnotesize\textbf{Abstract:} New York City’s MTA collaborated with Google to pilot an AI-based track monitoring tool called \textit{TrackInspect}. Off-the-shelf Google Pixel smartphones were mounted inside subway cars to continuously collect motion and sound data using built-in sensors like accelerometers, gyroscopes, and microphones. These sensor readings were processed by machine learning algorithms to detect anomalies in track conditions. Over the trial period, the system successfully identified about 92\% of the track defects later confirmed by human inspectors, demonstrating that consumer-grade hardware combined with computing intelligence can be a powerful tool in infrastructure maintenance.\\

 \item \textit{University of Houston News (2025)} -- \href{https://technews.acm.org/archives.cfm?fo=2025-04-apr#41}{\textbf{Wearable Pediatric Soft Exoskeleton Made of Smart Materials}}\\
\footnotesize\textbf{Abstract:} Engineers developed MyoStep, a wearable exoskeleton designed for children with cerebral palsy. The device uses soft actuators and sensors to detect motion and provide assistive force to help with walking. It dynamically adjusts to the child’s gait and growth, enabling physical therapy and mobility improvements in real time.\\

 \item \textit{The Engineer (UK) (2024)} -- \href{https://technews.acm.org/archives.cfm?fo=2024-06-jun#48}{\textbf{Digital Twins Used to Improve Built Environments for Robots}}\\
\footnotesize\textbf{Abstract:} Researchers are building digital twins—virtual simulations of real environments—to test robot deployment. They scan and map building features, simulate robot behavior in the digital space, then suggest modifications to improve robot performance. This three-phase system combines spatial data, simulation modeling, and architectural feedback.\\

    % Add more items here in the same format
\end{enumerate}

\vfill
\noindent \textit{Optional: What assumptions or concerns did you notice about how this system works? Who might it not work for? What could go wrong if the contract fails?}

\end{document}



