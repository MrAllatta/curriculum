\documentclass[11pt]{article}
\usepackage{geometry}
\geometry{margin=1in}
\usepackage{titlesec}
\usepackage{enumitem}
\usepackage{hyperref}
\usepackage{parskip}
\usepackage{fancyhdr}
\pagestyle{fancy}
\fancyhf{}
\rhead{Intro to CS – Museum School}
\lhead{2025}

\titleformat{\section}{\large\bfseries}{}{0em}{}
\titleformat{\subsection}{\normalsize\bfseries}{}{0em}{}

\begin{document}

\begin{center}
\LARGE Programming by Design: Computing, Representation, and Reasoning \\
\normalsize 9th Grade Computer Science – NYC Museum School \\
\vspace{0.5em}
\end{center}

\section*{Course Overview}
This yearlong course introduces students to computer science as a discipline of design, logic, and representation. Through structured, inquiry-driven modules, students learn to write programs, model systems, and interpret data. Programming is framed as structured communication between humans and machines.

Students will work across languages—beginning in Racket, transitioning to Pyret, and moving toward Python—building transferable fluency and computational reasoning. Each unit grounds technical skill in social context, particularly through museum-based datasets and themes.

\section*{Units of Study}

\subsection*{Unit 1: The Story of Data}
\begin{itemize}[noitemsep]
  \item Communication between humans and machines
  \item Histories of Ada Lovelace, Grace Hopper, and underrepresented computing figures
  \item Machine code, abstraction, and the need for high-level languages
  \item Weekly routines: computing in the news, visual data interpretation
\end{itemize}

\subsection*{Unit 2: Programming by Design (Racket)}
\begin{itemize}[noitemsep]
  \item Structured problem-solving with design recipes
  \item Algebraic reasoning, subgoal decomposition, function testing
  \item Build reusable, interpretable programs from first principles
\end{itemize}

\subsection*{Unit 3: Data Science and Representation (Pyret)}
\begin{itemize}[noitemsep]
  \item Explore, clean, and transform real datasets (e.g. Met API)
  \item Model categories, visualize patterns, critique bias in data
  \item Compare languages: Racket to Pyret transfer
\end{itemize}

\subsection*{Unit 4: Systems and Control}
\begin{itemize}[noitemsep]
  \item Conditionals, boolean logic, simulations
  \item Students build interactive systems (e.g., exhibit traffic simulation)
\end{itemize}

\subsection*{Unit 5: Interface and Presentation (HTML/CSS)}
\begin{itemize}[noitemsep]
  \item Web literacy and front-end structures
  \item Publish student work with accessible, styled interfaces
\end{itemize}

\subsection*{Unit 6: Code in the Wild (APIs + Notebooks)}
\begin{itemize}[noitemsep]
  \item Introduction to Jupyter notebooks and literate computing
  \item Fetching and analyzing data via APIs
  \item Scaffold toward Python and advanced CS courses
\end{itemize}

\subsection*{Ethics Thread (Interwoven)}
\begin{itemize}[noitemsep]
  \item Algorithmic bias, surveillance, equity in data systems
  \item Reflection on computing's civic, ethical, and cultural roles
\end{itemize}

\section*{Pathways Enabled}
\begin{itemize}[noitemsep]
  \item \textbf{AP CS A}: Structural foundations prepare students for Java and OOP
  \item \textbf{Data Science + Capstone}: Jupyter, visualization, and museum data carry into inquiry-based capstone work
  \item \textbf{Algebra 2 + Computing}: Reinforces function structure and modeling as algebraic tools
\end{itemize}

\section*{Select Bibliography}
\begin{itemize}[noitemsep]
  \item Felleisen et al., \textit{How to Design Programs}
  \item Bootstrap: Algebra and Data Science Curriculum
  \item \textit{Data Feminism} – D'Ignazio and Klein
  \item Exploring Computer Science (ECS)
  \item Metropolitan Museum of Art Open Access Data
  \item NYT Learning Network – \textit{What's Going On In This Graph?}
\end{itemize}

\end{document}
