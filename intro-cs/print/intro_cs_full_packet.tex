\documentclass[11pt]{article}
\usepackage[margin=1in]{geometry}
\usepackage{titlesec}
\usepackage{enumitem}
\usepackage{parskip}
\usepackage{fancyhdr}
\usepackage{hyperref}
\usepackage{lmodern}

\pagestyle{fancy}
\fancyhf{}
\lhead{NYC Museum School}
\rhead{Programming by Design}
\cfoot{\thepage}

\titleformat{\section}{\Large\bfseries}{}{0em}{}
\titleformat{\subsection}{\normalsize\bfseries}{}{0em}{}

\begin{document}

\begin{center}
  {\LARGE \textbf{Programming by Design}} \\[0.25em]
  {\large \textit{Computing, Representation, and Reasoning}} \\[0.5em]
  \textbf{Grade Level: 9th Grade \quad | \quad Format: Modular, Project-Based Yearlong Course} \\[0.5em]
  \textit{Instructor: Eric Allatta}
\end{center}

\section*{Course Overview}
This full-year course introduces computer science as a discipline of structured reasoning, communication, and system design. Students explore computing as both a technical and cultural practice: one grounded in logic and abstraction, but inseparable from authorship, ethics, and representation.

Rather than start with syntax, students begin by investigating how structured thinking allows us to build systems, interpret information, and design solutions. They move from models and metaphors to code and communication\textemdash building fluency across representations and reflecting on what it means to \textquotedblleft think computationally.\textquotedblright

\textbf{Guiding Principles:}
\begin{itemize}[leftmargin=*]
  \item Programming is a language for structured thinking
  \item Data is never neutral\textemdash it must be interpreted and questioned
  \item Systems must be tested, visualized, and explained
  \item Good code is communicative: to machines \textit{and} to people
  \item Fluency comes from routine: structured practice in writing, naming, diagramming, and revising
  \item Tools are thinking tools: CLI, markdown, version control, and diagrams support student authorship and structure
\end{itemize}

\section*{Conceptual Arc}
\begin{center}
\begin{tabular}{|l|l|l|}
  \hline
  \textbf{Unit} & \textbf{Focus} & \textbf{Tool/Language} \\
  \hline
  0 & Orientation: The Story of Data & Systems + Representation (no code) \\
  1 & Functional Design & Racket \\
  2 & Data Science + Representation & Pyret \\
  3 & Control + State & Python + EarSketch \\
  4 & Interfaces + Interpretation & HTML/CSS \\
  5 & Inquiry + Real Data & APIs + Jupyter \\
  6 & Infrastructure + Power & Internet Simulator + CLI \\
  7 & Capstone & Student Choice \\
  \hline
\end{tabular}
\end{center}

\section*{Strategic Pathways}
\textbf{1. AP Computer Science A}
\begin{itemize}[leftmargin=*]
  \item Emphasis on control structures, decomposition, and algorithmic reasoning
  \item Strong foundation for Java syntax and object-oriented thinking
\end{itemize}

\textbf{2. Data Science + Capstone}
\begin{itemize}[leftmargin=*]
  \item Literate computing using notebooks and real-world data
  \item Emphasis on communication, reproducibility, and inquiry
\end{itemize}

\textbf{3. Algebra 2 + Computing Integration}
\begin{itemize}[leftmargin=*]
  \item Function modeling, conditional logic, and recursion
  \item Strong cross-disciplinary links between CS and math reasoning
\end{itemize}

\newpage

\section*{Appendix: Selected Unit Summaries}

\subsection*{Unit 0: The Story of Data \textemdash Orientation Through Structure and Representation}
\textbf{Framing Concept:} Computing is a human system of structured communication.

This is a non-coding unit. Students begin with the intellectual and cultural foundations of computer science\textemdash exploring computation as a set of human decisions about meaning, structure, and systems. They practice thinking precisely, reflecting honestly, and diagramming systems that shape their lives.

\textbf{Key Outcomes:}
\begin{itemize}[leftmargin=*]
  \item Understand computing as a system of structured representation
  \item Build fluency with core metaphors (systems, abstraction, encoding, authorship)
  \item Normalize error, reflection, and identity as part of learning
  \item Establish durable routines: journals, diagrams, and prompt logs
\end{itemize}

\subsection*{Unit 1: Programming by Design (Racket)}
\textbf{Framing Concept:} A program is a structured solution to a problem.

Students use Racket to design pure functions using the design recipe: contract, examples, definition, and tests. They build recursive and piecewise functions and apply them to visual and mathematical models.

\textbf{Key Outcomes:}
\begin{itemize}[leftmargin=*]
  \item Write and explain testable functions
  \item Practice decomposition and reuse
  \item Begin recursive reasoning aligned with algebraic logic
\end{itemize}

\subsection*{Unit 2: Data Science and Representation (Pyret)}
\textbf{Framing Concept:} Data is a constructed lens on the world.

Students manipulate tabular datasets to explore representation and omission. They apply \texttt{filter}, \texttt{map}, and \texttt{build-column} functions to real museum or cultural datasets. Data visualizations support critical interpretation.

\textbf{Key Outcomes:}
\begin{itemize}[leftmargin=*]
  \item Use Boolean expressions and functions to analyze data
  \item Create and critique graphs and transformations
  \item Explore the role of categorization and metadata in shaping knowledge
\end{itemize}

\subsection*{Unit 3: Systems and Control (Python + EarSketch)}
\textbf{Framing Concept:} Programs model dynamic systems through control flow and state.

Students shift into procedural thinking using loops, conditionals, and accumulators. In EarSketch, they create rule-based sound compositions; in vanilla Python, they simulate behaviors and interactive systems.

\textbf{Key Outcomes:}
\begin{itemize}[leftmargin=*]
  \item Write code using \texttt{for}, \texttt{while}, \texttt{if/else} constructs
  \item Understand mutation, state, and behavior
  \item Trace and debug procedural logic with intention
\end{itemize}

\subsection*{Capstone Preview}
The course concludes with an independent capstone. Students synthesize learning through a data inquiry, system simulation, or interface project.

\textbf{Deliverables:}
\begin{itemize}[leftmargin=*]
  \item Final artifact (code, visualization, or site)
  \item Technical documentation + logic explanation
  \item Reflective writing and presentation
\end{itemize}

\end{document}
