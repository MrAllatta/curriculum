\documentclass[11pt]{article}
\usepackage[margin=1in]{geometry}
\usepackage{titlesec}
\usepackage{enumitem}
\usepackage{parskip}
\usepackage{fancyhdr}
\usepackage{lmodern}
\usepackage{hyperref}

\pagestyle{fancy}
\fancyhf{}
\lhead{NYC Museum School}
\rhead{CS9 Capstone Project}
\cfoot{\thepage}

\titleformat{\section}{\Large\bfseries}{}{0em}{}
\titleformat{\subsection}{\normalsize\bfseries}{}{0em}{}

\begin{document}

\begin{center}
  {\LARGE \textbf{CS9 Capstone Project Packet}} \\[0.25em]
  {\large Programming by Design: Computing, Representation, and Reasoning} \\[0.5em]
  \textbf{Instructor: Eric Allatta \quad | \quad Grade Level: 9th Grade}
\end{center}

\section*{Overview}
The Capstone Project is the culminating experience of the course. It is an opportunity for you to demonstrate fluency in reasoning, communication, and computation by designing a final project that reflects your growth.

This is not a project for flash or gimmicks. It is a project that shows you can structure a problem, design a system, build it thoughtfully, and explain your choices clearly.

\section*{Capstone Pathways}
Students may choose one of the following tracks or propose a hybrid project with approval.

\subsection*{1. Data Narrative}
\begin{itemize}[leftmargin=*]
  \item Pose a meaningful question using a real dataset
  \item Clean, filter, and transform data to investigate the question
  \item Visualize findings using graphs, charts, or computed columns
  \item Document your logic and data decisions in narrative form
\end{itemize}

\subsection*{2. System Simulation}
\begin{itemize}[leftmargin=*]
  \item Model a rule-based process using loops and conditionals
  \item Simulate dynamic behavior (e.g., population, traffic, interaction)
  \item Design input/output relationships and explain control flow
  \item Emphasize logic correctness and clarity
\end{itemize}

\subsection*{3. Interface and Interpretation}
\begin{itemize}[leftmargin=*]
  \item Build a static or interactive site to explain a CS topic or showcase prior work
  \item Focus on communication, access, and presentation
  \item Integrate HTML/CSS and optional JavaScript or embedded visuals
  \item Make your audience central to design choices
\end{itemize}

\section*{Required Deliverables}
\begin{itemize}[leftmargin=*]
  \item \textbf{Final Artifact:} code, simulation, notebook, or site
  \item \textbf{Technical Documentation:} purpose, logic design, key decisions
  \item \textbf{Reflection:} What you built, how you built it, and what you learned
  \item \textbf{Peer Review Checkpoint:} give and receive meaningful critique
\end{itemize}

\section*{Milestones + Timeline}
\begin{itemize}[leftmargin=*]
  \item \textbf{Week 1:} Project pitch + approval
  \item \textbf{Week 2:} Build draft + checkpoint feedback
  \item \textbf{Week 3:} Finalize and document project
  \item \textbf{Week 4:} Presentations and peer reflections
\end{itemize}

\section*{Presentation + Evaluation}
Projects will be evaluated using the following criteria:
\begin{itemize}[leftmargin=*]
  \item \textbf{Clarity:} Can someone understand what it does and why?
  \item \textbf{Structure:} Does it reflect strong logic and clean design?
  \item \textbf{Explanation:} Can you defend your choices and method?
  \item \textbf{Coherence:} Does everything fit together meaningfully?
  \item \textbf{Polish:} Is it complete, accurate, and visually considered?
\end{itemize}

\section*{Student Checklist}
\begin{itemize}[leftmargin=*]
  \item [ ] I can explain my project's purpose clearly
  \item [ ] I have documented my logic and structure
  \item [ ] I sought and used feedback to revise
  \item [ ] My project is complete and communicates effectively
  \item [ ] I am prepared to present and respond to questions
\end{itemize}

\end{document}
