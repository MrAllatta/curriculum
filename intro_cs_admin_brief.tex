\documentclass[11pt]{article}
\usepackage[margin=1in]{geometry}
\usepackage{titlesec}
\usepackage{enumitem}
\usepackage{parskip}
\usepackage{fancyhdr}
\usepackage{lmodern}

\pagestyle{fancy}
\fancyhf{}
\lhead{NYC Museum School}
\rhead{CS9 Admin Brief}
\cfoot{\thepage}

\titleformat{\section}{\Large\bfseries}{}{0em}{}
\titleformat{\subsection}{\normalsize\bfseries}{}{0em}{}

\begin{document}

\begin{center}
  {\LARGE \textbf{Admin Brief}} \\[0.25em]
  {\large Programming by Design: Computing, Representation, and Reasoning} \\[0.5em]
  \textbf{Grade 9 Computer Science \quad | \quad Instructor: Eric Allatta}
\end{center}

\section*{Course Summary}
This course introduces students to computer science as a discipline of structured reasoning, communication, and systems thinking. It is not a coding course. It is a course where students learn how to decompose problems, model ideas precisely, and communicate their thinking through code.

Students begin with logic and functional design, build through data analysis and simulation, and conclude with applied systems thinking, internet architecture, and a student-directed capstone. Each unit is modular, inquiry-driven, and aligned to real-world tools and conceptual fluency.

\section*{Design Priorities}
\begin{itemize}[leftmargin=*]
  \item \textbf{Structure Before Syntax:} Students start with clarity of thought before navigating procedural noise.
  \item \textbf{Fluency Across Representations:} From functions to data to control structures, students transfer ideas across languages.
  \item \textbf{Threaded Ethics + Interpretation:} Students engage equity, bias, and infrastructure as ongoing threads—not isolated topics.
  \item \textbf{Authentic Tools + Inquiry:} Students use Pyret, Python, HTML/CSS, APIs, and Jupyter Notebooks in contextually meaningful ways.
\end{itemize}

\section*{Strategic Outcomes}
\begin{itemize}[leftmargin=*]
  \item \textbf{AP CS A Readiness:} Foundational fluency with control flow, decomposition, and testing.
  \item \textbf{Data Science + Capstone:} Inquiry-based work with real datasets, documentation, and ethical analysis.
  \item \textbf{Algebra Integration:} Functions, conditional reasoning, and structural logic aligned to Algebra 2 outcomes.
\end{itemize}

\section*{Key Instructional Features}
\begin{itemize}[leftmargin=*]
  \item \textbf{Routines:} Weekly graph interpretation, journal reflection, computing in the news.
  \item \textbf{Assessment:} Emphasis on clarity, structure, and explanation—not speed or gimmicks.
  \item \textbf{Capstone Project:} Culminating student-designed artifact demonstrating reasoning and communication.
  \item \textbf{Tools:} Racket, Pyret, Python, HTML/CSS, APIs, Jupyter, CLI (optional).
\end{itemize}

\section*{Implementation Notes}
\begin{itemize}[leftmargin=*]
  \item Modular units allow for differentiation, pacing variation, or adaptation into interdisciplinary teams.
  \item Materials are aligned to Bootstrap, Math for America, and existing NYC CS/Math integration frameworks.
  \item Assessment structures support student-led growth: rubric-based, reflective, and publicly presentable.
\end{itemize}

\section*{Contact}
For questions, materials, or full syllabus access, contact: \\[0.25em]
\textbf{Eric Allatta} — Curriculum Lead \\ Email: [your.email@domain.com]

\end{document}
