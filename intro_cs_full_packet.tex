\documentclass[11pt]{article}
\usepackage[margin=1in]{geometry}
\usepackage{titlesec}
\usepackage{enumitem}
\usepackage{parskip}
\usepackage{fancyhdr}
\usepackage{hyperref}
\usepackage{lmodern}

\pagestyle{fancy}
\fancyhf{}
\lhead{NYC Museum School}
\rhead{Programming by Design}
\cfoot{\thepage}

\titleformat{\section}{\Large\bfseries}{}{0em}{}
\titleformat{\subsection}{\normalsize\bfseries}{}{0em}{}

\begin{document}

\begin{center}
  {\LARGE \textbf{Programming by Design}} \\[0.25em]
  {\large \textit{Computing, Representation, and Reasoning}} \\[0.5em]
  \textbf{Grade Level: 9th Grade \quad | \quad Format: Modular, Project-Based Yearlong Course} \\[0.5em]
  \textit{Instructor: Eric Allatta}
\end{center}

\section*{Course Overview}
This course treats computer science as a discipline of design, logic, and interpretation—not just syntax or application. Students begin with structure, build through data, and extend to systems thinking, human-computer interaction, and inquiry through code.

\textbf{Guiding Principles:}
\begin{itemize}[leftmargin=*]
  \item Programming is a language for structured thinking
  \item Data is never neutral—it must be interpreted and questioned
  \item Systems must be tested, visualized, and explained
  \item Good code is communicative: to machines \textit{and} to people
\end{itemize}

\section*{Conceptual Arc}
\begin{center}
\begin{tabular}{|l|l|l|}
  \hline
  \textbf{Unit} & \textbf{Focus} & \textbf{Tool/Language} \\
  \hline
  1 & Computing as Communication & Concepts + History \\
  2 & Functional Design & Racket \\
  3 & Data Science + Representation & Pyret \\
  4 & Control + State & Python + EarSketch \\
  5 & Interfaces + Interpretation & HTML/CSS \\
  6 & Inquiry + Real Data & APIs + Jupyter \\
  7 & Infrastructure + Power & Internet Simulator + CLI \\
  8 & Capstone & Student Choice \\
  \hline
\end{tabular}
\end{center}

\section*{Strategic Pathways}
\textbf{1. AP Computer Science A}
\begin{itemize}[leftmargin=*]
  \item Emphasis on control structures, decomposition, and algorithmic reasoning
  \item Strong foundation for Java syntax and object-oriented thinking
\end{itemize}

\textbf{2. Data Science + Capstone}
\begin{itemize}[leftmargin=*]
  \item Literate computing using notebooks and real-world data
  \item Emphasis on communication, reproducibility, and inquiry
\end{itemize}

\textbf{3. Algebra 2 + Computing Integration}
\begin{itemize}[leftmargin=*]
  \item Function modeling, conditional logic, and recursion
  \item Strong cross-disciplinary links between CS and math reasoning
\end{itemize}

\newpage

\section*{Appendix: Selected Unit Summaries}

\subsection*{Unit 1: The Story of Data}
\textbf{Framing Concept:} Programming is communication between humans and machines.

Students begin with the epistemological and historical foundations of computing. Through readings, visual mapping, and structured discussion, they engage with the idea that computation is designed, contextual, and cultural. Activities include human-algorithm exercises, computing in the news, and interpretive graph reading.

\textbf{Key Outcomes:}
\begin{itemize}[leftmargin=*]
  \item Understand data as structured representation
  \item Build foundational metaphors (system, abstraction, encoding)
  \item Introduce equity and underrepresentation in computing history
\end{itemize}

\subsection*{Unit 2: Programming by Design (Racket)}
\textbf{Framing Concept:} A program is a structured solution to a problem.

Students use Racket to design pure functions using the design recipe: contract, examples, definition, and tests. They build recursive and piecewise functions and apply them to visual and mathematical models.

\textbf{Key Outcomes:}
\begin{itemize}[leftmargin=*]
  \item Write and explain testable functions
  \item Practice decomposition and reuse
  \item Begin recursive reasoning aligned with algebraic logic
\end{itemize}

\subsection*{Unit 3: Data Science and Representation (Pyret)}
\textbf{Framing Concept:} Data is a constructed lens on the world.

Students manipulate tabular datasets to explore representation and omission. They apply \texttt{filter}, \texttt{map}, and \texttt{build-column} functions to real museum or cultural datasets. Data visualizations support critical interpretation.

\textbf{Key Outcomes:}
\begin{itemize}[leftmargin=*]
  \item Use Boolean expressions and functions to analyze data
  \item Create and critique graphs and transformations
  \item Explore the role of categorization and metadata in shaping knowledge
\end{itemize}

\subsection*{Unit 4: Systems and Control (Python + EarSketch)}
\textbf{Framing Concept:} Programs model dynamic systems through control flow and state.

Students shift into procedural thinking using loops, conditionals, and accumulators. In EarSketch, they create rule-based sound compositions; in vanilla Python, they simulate behaviors and interactive systems.

\textbf{Key Outcomes:}
\begin{itemize}[leftmargin=*]
  \item Write code using \texttt{for}, \texttt{while}, \texttt{if/else} constructs
  \item Understand mutation, state, and behavior
  \item Trace and debug procedural logic with intention
\end{itemize}

\subsection*{Capstone Preview}
The course concludes with an independent capstone. Students synthesize learning through a data inquiry, system simulation, or interface project.

\textbf{Deliverables:}
\begin{itemize}[leftmargin=*]
  \item Final artifact (code, visualization, or site)
  \item Technical documentation + logic explanation
  \item Reflective writing and presentation
\end{itemize}

\end{document}
